\begin{document}
Nach der Analyse von verschiedenen Visualisierungsmethodiken musste der Umfang und die Komplexität der Applikation bestimmt werden.
Folgende Kriterien wurden festgelegt:
\begin{itemize}
    \item Visualisierung sollte unspezifisch auf Daten funktionieren
    \item Benutzer sollte in der Lage sein, die Struktur des Grafen selbst zu bestimmen
    \item 3D Welt sollte ohne weitere Benutzer Interaktion generiert werden
\end{itemize}
Des weiteren wurde festgelegt, das die Webseite mit dem Framework ReactJS und der Server der die Webseite veröffentlichen würde mit NodeJS geschrieben
wird.

\section{Server}
Die Entscheidung einen Server zu schreiben hat mehrere Gründe auf die ich in diesem Kapitel kurz eingehen werde. \\ \\
Einer der Gründe ist das die Anwendung über einen Server veröffentlicht werden muss um im besten Fall diesen auf einen herkömmlichen Server im
Internet hochladen zu können. Ein weiterer Grund war, das in dem Design Prozess die Idee existierte, einen MongoDB Server über Docker Instanz
zu verwalten. Dieser würde die Datenstruktur, die Ausgewählt und Generiert wurde, abspeichern und zur abfrage jederzeit bereit halten. \\ \\
Wie sich später auch herausstelle wäre der finale Grund den Server zu verwenden, das Browser einen lokalen Cache Limit haben. Somit können
größere Dateien (500 MB und größer) nicht mehr direkt vom Browser verarbeitet werden. Diese müssten in einer produktiv Umgebung erstmal auf
den Server hochgeladen. Dort analysiert, verarbeitet und umgewandelt werden, und zu guter Letzt wieder zu der Anwendung zurück geschickt
werden.
\newpage
\section{Applikation}
Basierend auf den Kriterien wurde die Funktionalität der Anwendung in drei Abschnitte unterteilt.
\begin{itemize}
    \item Anzeige \& Konfiguration
    \item Analyse
    \item Welt Generierung
\end{itemize}
Zu dem musste ich mir in diesem Stadium schon Gedanken machen wie meine Daten in der Praxis aussehen würden also wurde Datenquellen wie
imdb angeschaut. Worauf hin fest gestellt wurde das der größte Teil von solchen Daten im CSV, TSV oder JSON Format ausgehändigt werden. Mit
diesem Wissen ging ich an die Logik ran diese Daten anzuzeigen und zu konfigurieren.

\subsection{Anzeige \& Konfiguration}
Die Anzeige \& Konfiguration der Daten wird wiederum über drei Schritte absolviert. Zuerst muss eine valide csv, tsv oder json (json in einer
flachen Hierarchie) ausgewählt werden.
\begin{lstlisting}
                                Beispiel JSON
{
    {
        id: "0",
        title: "Geschichte des VR",
        author: "Walter Guenther",
        veroeffentlicht: "1996"
    },
    {
        id: "1",
        title: "Datenvisualisierung 101",
        author: "Peter Watson",
        veroeffentlicht: "2001"
    }
}
\end{lstlisting}
Diese Datei verarbeite ich un visualiese dem Benuzter was für ein Inhalt er in der Datei hat und wie Daten ungefähr aussehen.
% take screenshot of application here
%\begin{center}
    %\includegraphcis[width=0.5\textwidth]{}
%\end{center}
\end{document}
