\begin{document}
Nach der Analyse von verschiedenen Visualisierungsmethodiken musste der Umfang und die Komplexität der Applikation bestimmt werden.
Folgende Kriterien wurden festgelegt:
\begin{itemize}
    \item Visualisierung sollte unspezifisch auf Daten funktionieren
    \item Benutzer sollte in der Lage sein, die Struktur des Grafen selbst zu bestimmen
    \item 3D Welt sollte ohne weitere Benutzer Interaktion generiert werden
\end{itemize}
Des weiteren wurde festgelegt, das die Webseite mit dem Framework ReactJS und der Server der die Webseite veröffentlichen würde mit NodeJS geschrieben
wird.

\section{Server}
Die Entscheidung einen Server zu schreiben hat mehrere Gründe auf die ich in diesem Kapitel kurz eingehen werde. \\ \\
Einer der Gründe ist das die Anwendung über einen Server veröffentlicht werden muss um im besten Fall diesen auf einen herkömmlichen Server im
Internet hochladen zu können. Ein weiterer Grund war, das in dem Design Prozess die Idee existierte, einen MongoDB Server über Docker Instanz
zu verwalten. Dieser würde die Datenstruktur, die Ausgewählt und Generiert wurde, abspeichern und zur abfrage jederzeit bereit halten. \\ \\
Wie sich später auch herausstelle wäre der finale Grund den Server zu verwenden, das Browser einen lokalen Cache Limit haben. Somit können
größere Dateien (500 MB und größer) nicht mehr direkt vom Browser verarbeitet werden. Diese müssten in einer produktiv Umgebung erstmal auf
den Server hochgeladen. Dort analysiert, verarbeitet und umgewandelt werden, und zu guter Letzt wieder zu der Anwendung zurück geschickt
werden.

\section{Applikation}

\end{document}
